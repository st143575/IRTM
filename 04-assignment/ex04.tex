\documentclass[a4paper]{scrartcl}
\usepackage{makecell}
\usepackage{multicol}
\usepackage{graphicx}
\usepackage{anysize}
\usepackage{amsmath}
\usepackage{kpfonts}
\usepackage{tabularx}
\usepackage{hyperref}
\usepackage{listings}
\usepackage{color}
\usepackage[parfill]{parskip}

\marginsize{25mm}{25mm}{25mm}{25mm}
%---Code-Editor Config ------------------------%
\definecolor{dkgreen}{rgb}{0,0.6,0}
\definecolor{gray}{rgb}{0.5,0.5,0.5}
\definecolor{mauve}{rgb}{0.58,0,0.82}
\definecolor{backcolour}{rgb}{0.95,0.95,0.92}
\lstset{basicstyle=\ttfamily}
\lstset{literate=%
    {á}{{\'a}}1 {é}{{\'e}}1 {í}{{\'i}}1 {ó}{{\'o}}1 {ú}{{\'u}}1
    {Á}{{\'A}}1 {É}{{\'E}}1 {Í}{{\'I}}1 {Ó}{{\'O}}1 {Ú}{{\'U}}1
	{à}{{\`a}}1 {è}{{\`e}}1 {ì}{{\`i}}1 {ò}{{\`o}}1 {ù}{{\`u}}1
	{À}{{\`A}}1 {È}{{\'E}}1 {Ì}{{\`I}}1 {Ò}{{\`O}}1 {Ù}{{\`U}}1
	{ä}{{\"a}}1 {ë}{{\"e}}1 {ï}{{\"i}}1 {ö}{{\"o}}1 {ü}{{\"u}}1
	{Ä}{{\"A}}1 {Ë}{{\"E}}1 {Ï}{{\"I}}1 {Ö}{{\"O}}1 {Ü}{{\"U}}1
	{â}{{\^a}}1 {ê}{{\^e}}1 {î}{{\^i}}1 {ô}{{\^o}}1 {û}{{\^u}}1
	{Â}{{\^A}}1 {Ê}{{\^E}}1 {Î}{{\^I}}1 {Ô}{{\^O}}1 {Û}{{\^U}}1
	{œ}{{\oe}}1 {Œ}{{\OE}}1 {æ}{{\ae}}1 {Æ}{{\AE}}1 {ß}{{\ss}}1
	{ç}{{\c c}}1 {Ç}{{\c C}}1 {ø}{{\o}}1 {å}{{\r a}}1 {Å}{{\r A}}1
	{€}{{\EUR}}1 {£}{{\pounds}}1
	{>}{{$>$}}1{<}{{$<$}}1
}
\lstset{
	language=Python,				% the language of the code
	basicstyle=\footnotesize,			% the size of the fonts that are used for the code
	numbers=left,					% where to put the line-numbers
	numberstyle=\tiny\color{gray},		% the style that is used for the line-numbers
	stepnumber=1,					% the step between two line-numbers. If it's 1, each line will be numbered
	numbersep=5pt,				% how far the line-numbers are from the code
	backgroundcolor=\color{white},		% choose the background color. You must add \usepackage{color}
	showspaces=false,				% show spaces adding particular underscores
	showstringspaces=false,			% underline spaces within strings
	showtabs=false,				% show tabs within strings adding particular underscores
	frame=single,					% adds a frame around the code
	rulecolor=\color{black},			% if not set, the frame-color may be changed on line-breaks within not-black text (e.g. commens (green here))
	tabsize=2,						% sets default tabsize to 2 spaces
	captionpos=b,					% sets the caption-position to bottom
	breaklines=true,                			% sets automatic line breaking
  	breakatwhitespace=false,       		% sets if automatic breaks should only happen at whitespace
  	title=\lstname,        % show the filename of files included with \lstinputlisting; % also try caption instead of title
  	keywordstyle=\color{blue},          	% keyword style
  	commentstyle=\color{dkgreen},       	% comment style
  	stringstyle=\color{mauve},         		% string literal style
  	escapeinside={\%*}{*)},            		% if you want to add LaTeX within your code
  	morekeywords={*,...}              		% if you want to add more keywords to the set
}

%Header and Footer -----------------------%
\usepackage[headsepline]{scrlayer-scrpage}
\pagestyle{scrheadings}
\clearpairofpagestyles
%\setlength{\headheight}{40.8pt}
\setlength{\headheight}{56pt}
\ihead{IRTM\\ Wi 20/21\\ Assigment 4} 
\ohead{
    Alberto Saponaro - saponaroalberto97@gmail.com\\
    Walter Väth - walter.vaeth@gmail.com\\
    Chong Shen - st143575@stud.uni-stuttgart.de\\
    Xin Pang - st145113@stud.uni-stuttgart.de
}
\ofoot{\pagemark}



%-----------------------------------------------%
%  BEGIN                                        %
%-----------------------------------------------%
\begin{document}
    
\section*{Task 1}
\textbf{Vocabulary} = \{ happy, new, year, holiday, term, starts, work, celebrations \}\\
$|$Vocabulary$|$ = 8

\subsubsection*{Prior:}
    $P(c_1) = \frac{3}{5} = 0.6$\\
    $P(c_2) = \frac{2}{5} = 0.4$ 

\subsubsection*{Posterior: (with Add-One-Smoothing)}
\begin{multicols}{2}
    $P(happy|c_1) = \frac{2 + 1}{5 + 8} = \frac{3}{13}$\\
    $P(new|c_1) = \frac{2 + 1}{5 + 8} = \frac{3}{13}$\\
    $P(year|c_1) = \frac{2 + 1}{5 + 8} = \frac{3}{13}$\\
    $P(holiday|c_1) = \frac{1 + 1}{5 + 8} = \frac{2}{13}$\\
    $P(term|c_1) = \frac{0 + 1}{5 + 8} = \frac{1}{13}$\\
    $P(starts|c_1) = \frac{0 + 1}{5 + 8} = \frac{1}{13}$\\
    $P(work|c_1) = \frac{0 + 1}{5 + 8} = \frac{1}{13}$\\
    $P(celebrations|c_1) = \frac{0 + 1}{5 + 8} = \frac{1}{13}$

    $P(happy|c_2) = \frac{0 + 1}{5 + 8} = \frac{1}{13}$\\
    $P(new|c_2) = \frac{0 + 1}{5 + 8} = \frac{1}{13}$\\
    $P(year|c_2) = \frac{0 + 1}{5 + 8} = \frac{1}{13}$\\
    $P(holiday|c_2) = \frac{0 + 1}{5 + 8} = \frac{1}{13}$\\
    $P(term|c_2) = \frac{1 + 1}{5 + 8} = \frac{2}{13}$\\
    $P(starts|c_2) = \frac{2 + 1} + 1{5 + 8} = \frac{3}{13}$\\
    $P(work|c_2) = \frac{1 + 1}{5 + 8} = \frac{2}{13}$\\
    $P(celebrations|c_2) = \frac{0 + 1}{5 + 8} = \frac{1}{13}$
\end{multicols}

$P(c_1|d) = \frac{3}{13} * \frac{3}{13} * \frac{3}{13} * \frac{1}{13} = \frac{27}{28561} = c_{map}$\\
$P(c_2|d) = \frac{1}{13} * \frac{1}{13} * \frac{1}{13} * \frac{1}{13} = \frac{1}{28561}$

The model assigned the class $c_1$ to the document.




\section*{Programming Task}



\subsubsection*{OUTPUT:}
\begin{lstlisting}
	
\end{lstlisting}


\end{document}