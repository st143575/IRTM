\documentclass[a4paper]{scrartcl}
\usepackage{makecell}
\usepackage{multicol}
\usepackage{graphicx}
\usepackage{anysize}
\usepackage{amsmath}
\usepackage{kpfonts}
\usepackage{tabularx}
\usepackage{hyperref}
\usepackage{listings}
\usepackage{color}

\marginsize{25mm}{25mm}{25mm}{25mm}
%---Code-Editor Config ------------------------%
\definecolor{dkgreen}{rgb}{0,0.6,0}
\definecolor{gray}{rgb}{0.5,0.5,0.5}
\definecolor{mauve}{rgb}{0.58,0,0.82}
\definecolor{backcolour}{rgb}{0.95,0.95,0.92}
\lstset{basicstyle=\ttfamily}
\lstset{literate=%
  {Ö}{{\"O}}1
  {Ä}{{\"A}}1
  {Ü}{{\"U}}1
  {ß}{{\ss}}1
  {ü}{{\"u}}1
  {ä}{{\"a}}1
  {ö}{{\"o}}1
  {é}{{\"AC}}1
  {€}{{\"AC}}1
}
\lstset{
	language=Python,				% the language of the code
	basicstyle=\footnotesize,			% the size of the fonts that are used for the code
	numbers=left,					% where to put the line-numbers
	numberstyle=\tiny\color{gray},		% the style that is used for the line-numbers
	stepnumber=1,					% the step between two line-numbers. If it's 1, each line will be numbered
	numbersep=5pt,				% how far the line-numbers are from the code
	backgroundcolor=\color{white},		% choose the background color. You must add \usepackage{color}
	showspaces=false,				% show spaces adding particular underscores
	showstringspaces=false,			% underline spaces within strings
	showtabs=false,				% show tabs within strings adding particular underscores
	frame=single,					% adds a frame around the code
	rulecolor=\color{black},			% if not set, the frame-color may be changed on line-breaks within not-black text (e.g. commens (green here))
	tabsize=2,						% sets default tabsize to 2 spaces
	captionpos=b,					% sets the caption-position to bottom
	breaklines=true,                			% sets automatic line breaking
  	breakatwhitespace=false,       		% sets if automatic breaks should only happen at whitespace
  	title=\lstname,        % show the filename of files included with \lstinputlisting; % also try caption instead of title
  	keywordstyle=\color{blue},          	% keyword style
  	commentstyle=\color{dkgreen},       	% comment style
  	stringstyle=\color{mauve},         		% string literal style
  	escapeinside={\%*}{*)},            		% if you want to add LaTeX within your code
  	morekeywords={*,...}              		% if you want to add more keywords to the set
}

%Header and Footer -----------------------%
\usepackage[headsepline]{scrlayer-scrpage}
\pagestyle{scrheadings}
\clearpairofpagestyles
%\setlength{\headheight}{40.8pt}
\setlength{\headheight}{56pt}
\ihead{IRTM\\ Wi 20/21\\ Assigment 1} 
\ohead{
    Alberto Saponaro - saponaroalberto97@gmail.com\\
    Walter Väth - walter.vaeth@gmail.com\\
    Chong Shen - st143575@stud.uni-stuttgart.de\\
    Xin Pang - Email
}
\ofoot{\pagemark}

%-----------------------------------------------%
%  BEGIN                                        %
%-----------------------------------------------%
\begin{document}

\section*{Task 1}
\begin{itemize}
	\item Tkey-value pairs with (termID, docID).
	\item The parser writes the output to the segment files.
	\item The location of the relevant segment files.
	\item The collected values (docIDs) for a given key (termID).
\end{itemize}
\subsection*{Discssion:}
The right amount of parsers depend on the size of the input and how it is splitted. 
Therefore it depends on how well the input can be splitted in splits with a reasonable size. Each parser should have a similar amount of data to work on.
If we look for example at the English alphabetic characters, we could decide to split 26 characters into equal parts, e.g. divide 26 by 4 or by 6.
For a well-balanced working speed each splitted input should be of the same size and the absolute number of splits shouldn't be too large for better
efficiency. Therefore one parser for each term would be inefficient while one parser overall wouldn't be useful because there is no need for splitting data.\\

The parsers write the segment files, where one file exists for each term partition. We need an inverter for each term partition, because the master
gives for each term another inverter.

\section*{Task 2}
Break the update document into several splits and assign the splits to idle computer which parses these splits and outputs partitions of terms. Inverter uses these partitions to write new postings lists and merge them with the existing.\\
Only use the idle computer so that the entire system will not slow down too much.\\

\section*{Task 3}
\subsection*{Subtask 3.1}
$k=10, b\thickapprox 0,5$
\subsection*{Subtask 3.2}
$M=100000$

\clearpage
\section*{Task 4}
$216_{10} = 11011000_{2}$ \\ 
\textbf{variable byte code: } \textcolor{red}{0}0000001 \textcolor{red}{1}1011000 \\
\textbf{gamma code: }\\
Offset: $1011000$ \\ lenght $7_{10} = 11111110_{1}$ \\ $\gamma$-code: $111111101011000$

\section*{Task 5}
\textbf{Sequence: }1111011000100110000


\pagebreak
\section*{Programming Task 2}
\subsection*{Subtask 2}
\lstinputlisting{code/script.py}

\end{document}