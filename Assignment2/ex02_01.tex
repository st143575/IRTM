\subsection*{Task1}

\begin{itemize}
    \item What information does the task description contain that the master gives to a parser?\\
    key-value pairs with (termID, docID)\\
    \item What information does the parser report back to the master upon completion of the task?\\
    The parser writes the output to the segment files\\
    \item What information does the task description contain that the master gives to an inverter?\\
    the location of the relevant segment files\\
    \item What information does the inverter report back to the master upon completion of the task?\\
    The collected values (docIDs) for a given key (termID)
\end{itemize}

\subsection*{Discussion questions}
The right amount of parsers depend on the size of the input and how it is splitted. Therefore it depends on how well the input can be splitted
in splits with a reasonable size. Each parser should have a similar amount of data to work on.
If we look for example at the English alphabetic characters, we could decide to split 26 characters into equal parts, e.g. divide 26 by 4 or by 6.
For a well-balanced working speed each splitted input should be of the same size and the absolute number of splits shouldn't be too large for better
efficiency. Therefore one parser for each term would be inefficient while one parser overall wouldn't be useful because there is no need for splitting data.\\

The parsers write the segment files, where one file exists for each term partition. We need an inverter for each term partition, because the master
gives for each term another inverter.\\
